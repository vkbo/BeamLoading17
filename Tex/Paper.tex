
%%
%%  Beam Loading Paper '17'
%% ***************************
%%  Veronica K. Berglyd Olsen
%%

\documentclass[aps,prstab,reprint,groupedaddress]{revtex4-1}

% ************************************************************************************************ %
\begin{document}
% ************************************************************************************************ %

\title{Emittance preservation of an electron bunch in a loaded quasi-linear plasma wakefield}

\author{Veronica K. Berglyd Olsen}
\email[]{v.k.b.olsen@cern.ch}

\author{Erik Adli}
\affiliation{University of Oslo, Oslo, Norway}

\author{Patric Muggli}
\affiliation{Max Planck Institute for Physics, Munich, Germany}
\affiliation{CERN, Geneva, Switzerland}

\date{\today}

\begin{abstract}
We investigate beam loading and emittance preservation for a high-charge electron beam being accelerated in quasi-linear plasma wakefield driven by a short proton beam. The structure of the wakefield is similar to that of a long, modulated proton beam. By selecting transverse and longitudinal electron beam parameters in order to  appropriately load of the wake, we show that the bulk of the electron beam can be accelerated without significant emittance growth.
\end{abstract}

\maketitle

\section[\label{S:I}]{Introduction}

- interested from SMI/AWAKE \\
- requirements for AWAKE Run 2 \\

\section[\label{S:M}]{Method}
%- explain analogy of short-bunch, SMI case, by referring to earlier Veronica-papers \\
%- simulation setup PIC/quickpic OS \\

Initial beam loading studies were done using the full PIC code Osiris \cite{fonseca_osiris:_2002}. these simulations were done using a short, pre.dodulated prton drive bunch. That is, we applied a clipped cosine function to the longitudinal density profile of the proton beam, with a period matching the wavelength, $\lambda_p$, of the plasma.

\subsection[\label{S:M:Setup}]{Simulationn Setup}



\section[\label{S:BL}]{Beam loading}

- discussion of the main physics and results; beam loading, bubble creation, emittance preservation \\

\section[\label{S:D}]{Discussion}
\section[\label{S:C}]{Conclusion}

- discussion of optimal electron beam parameters \\
- implications for AWAKE Run 2 \\

\bibliography{Bibliography}

% ************************************************************************************************ %
\end{document}
% ************************************************************************************************ %
