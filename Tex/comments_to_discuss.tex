%Patric general comments

%energy labels not consistent between figs 7, 9 and 8
%labels for the fractional charge is too complicated on Figs 7,8,9. it needs to be explained in the caption and/or in the text. "On the left axis we plot the fraction of that charge whose emittance remains .... "

Veronica: to add sufficient graph explanations in the captions

%Add scale for grey color on Fig. 2

Veronica: to add scale for grey color (color bar with p.e. density)

%Fig. 3 has a "v=c" and an arrow that are only on that figure ...

Veronica: to add.

% the same symbol was used for sigma_x proton beam and electron beam.  I added ",proton" to the proton parameters to separate.  Should we also add ",electron" to ebeam?  

Veronica: check for consistency of ",proton" , also consider a different way of descriptin the two beams (sigma_x,p and sigma_x,e is ok with me).

%It is not clear to me what the conclusions/outcome of the last section are ....

I still think this section is needed, both to make the paper more relevant for AWAKE and to avoid (valid) criticism that we only study one parameter set, and do not discuss at all the dependence of the result on parameters.  I slightly rewrote the last line in the section to highlight that this section illustrates the parameter dependence in this accelerating regime. n

"The optimal working point will depend on the application and must be studied for each case and is, as illustrated in this section, a trade off between bunch length, bunch charge and emittance preservation, as well as other parameters like the plasma density."

(Veronica: unless better ideas no need to take action, expect review)

%THERE IS NO EMITTANCE SHOWN OR NOTHING TO COMPARE TO ... WHAT TO DO????

I have rewriten the text to use Fig 5 to compare the emittance growth, and not Fig 6.

(Veronica: unless better ideas no need to take action, expect review)


%~ - maybe also something about the fact that 200pC in a 60µm bunch corresponds to a 1kA current, much larger than the
%~   p+ bunch current, because of the fact that the wakefields are driven my multiple bunches?

Good idea, but I did not find an obvious place to state it without rewrite.  When we talk about the SSM bunch train we have not discussed the ebeam parameters.  And when we have discuss ebeam parameters we have made effort to focus on the single bunch proton driver.  So I did not implement this.

(Veronica: unless better ideas no need to take action, expect review)


